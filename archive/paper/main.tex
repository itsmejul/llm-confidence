\documentclass{article}
\usepackage{booktabs}     % for \toprule, \midrule, \bottomrule
\usepackage{geometry}     % better margins
\usepackage{longtable}    % support long tables (if needed)
\usepackage{caption}      % better table captions
\usepackage{multirow}

\geometry{margin=1in}

\title{}
\date{}

\begin{document}

\maketitle

\section{Abstract}
A major issue with Large Language Models (LLMs) is their tendency to hallucinate incorrect answers.
For this reason, several metrics have been devised in the attempt to measure how certain a model is 
in its answer, or how likely a answer is correct. One such measure is the token distribution entropy.  
In this work, we propose a new measure, the \textbf{S}emantic \textbf{To}ken Top-\textbf{P} \textbf{S}imilarity
 (STOPS). We observe a moderate to strong inverse correlation between entropy and STOPS. Furthermore, we evaluate the 
 suitability of classifiers based on entropy and STOPS for predicting model answer accuracy. 
 We support all of our findings through extensive experiments conducted on a multitude of datasets, models and prompting approaches.

 \section{Introduction}

 \section{Related Work}

 \section{Methodology}

 \section{Results}


 \section{Correlation between token entropy and semantic similarity of tokens}
The entropy of token probability distributions has been proposed as a measure for LLM uncertainty.
We propose a new uncertainty measure, which is the average pairwise cosine similarity of the top-p tokens of a probability distribution.
In the following, we will show that there is a inverse relationship

\begin{tabular}{lrr} \toprule Model & Pearson \(r\) & Spearman \(r\) \\ \midrule Qwen3-8B & -0.37 & -0.39 \\ Mistral-7B-v0.1 & -0.50 & -0.67 \\ Llama-3.1-8B-Instruct & -0.57 & -0.70 \\ Llama-3.1-8B & -0.62 & -0.76 \\ deepseek-llm-7b-base & -0.42 & -0.65 \\ \bottomrule \end{tabular}
\begin{tabular}{llrrrr}
\toprule
 &  & Pearson r & Pearson p & Spearman r & Spearman p \\
Model & Dataset &  &  &  &  \\
\midrule
\multirow[t]{3}{*}{Qwen3-8B} & gsm8k & -0.37 & 0.00 & -0.39 & 0.00 \\
 & writingprompts & -0.38 & 0.00 & -0.41 & 0.00 \\
 & xsum & -0.37 & 0.00 & -0.43 & 0.00 \\
\cline{1-6}
\multirow[t]{3}{*}{Mistral-7B-v0.1} & gsm8k & -0.50 & 0.00 & -0.67 & 0.00 \\
 & writingprompts & -0.49 & 0.00 & -0.66 & 0.00 \\
 & xsum & -0.51 & 0.00 & -0.67 & 0.00 \\
\cline{1-6}
\multirow[t]{3}{*}{Llama-3.1-8B-Instruct} & gsm8k & -0.57 & 0.00 & -0.70 & 0.00 \\
 & writingprompts & -0.58 & 0.00 & -0.70 & 0.00 \\
 & xsum & -0.57 & 0.00 & -0.69 & 0.00 \\
\cline{1-6}
\multirow[t]{3}{*}{Llama-3.1-8B} & gsm8k & -0.62 & 0.00 & -0.76 & 0.00 \\
 & writingprompts & -0.61 & 0.00 & -0.75 & 0.00 \\
 & xsum & -0.63 & 0.00 & -0.76 & 0.00 \\
\cline{1-6}
\multirow[t]{3}{*}{deepseek-llm-7b-base} & gsm8k & -0.42 & 0.00 & -0.65 & 0.00 \\
 & writingprompts & -0.43 & 0.00 & -0.67 & 0.00 \\
 & xsum & -0.45 & 0.00 & -0.67 & 0.00 \\
\cline{1-6}
\bottomrule
\end{tabular}


\section{Future Work}
Especially for the accuracy analysis, it is still open to perform it on more datasets and perhaps 
find a dataset where the cosine is a stronger indicator than entropy.

\end{document}
% related work paper
%https://aclanthology.org/2024.eacl-long.129.pdf#:~:text=The%20term%20diversity%20often%20refers,t%29%2C%20x%3B%20%CE%B8